% The following conversation is generated by ChatGPT (with GPT-3.5).

Q:
  对调查问卷进行判断,已判断哪些回答是认真的,哪些可能的敷衍。

A:
  此调查问卷包含关于学生对于人工智能相关课程支持、企业提供的课程资源、合作提供的资源和活动的期待程度,以及对实习或实训项目关心的因素等多个方面的问题。调查问卷还询问了学生对实验室、实践基地的实习实训的期望以及对提高人工智能教育质量和实践教学的其他意见和建议。

为了分析回答问题的时间以及判断哪些回答是认真的,哪些可能是敷衍,考虑以下几个方面:

\begin{itemize}
\item 回答时间长度:完成问卷的总时间,可以通过“开始时间”和“完成时间”计算得出。时间太短可能表明回答是敷衍的。

\item 答案的详细程度:可以通过检查开放式问题的答案长度和内容的详细程度来判断。答案内容丰富、具体的往往更可能是认真回答。

\item 答案的多样性:对于选择题,如果答案包含多个选项,可能表明回答者认真考虑了问题。单一答案可能不够充分。
\end{itemize}

首先计算每个回答的完成时间,然后分析开放式问题的答案长度,以此来判断哪些回答可能是认真的,哪些可能是敷衍的。

回答时间的统计分析显示:

\begin{itemize}

\item 平均回答时间为201.36秒(约3.4分钟)。
\item 回答时间的中位数为166秒(约2.8分钟)。

\item 回答时间最短为37秒,最长为1203秒(约20分钟)。
\item 大部分回答(75\%)在242.5秒(约4分钟)内完成。
\end{itemize}

Q:
考虑开放式问题答案长度因素。

A:
开放式问题答案长度的统计分析显示:

\begin{itemize}
\item 对于“您对实验室、实践基地的实习实训有何期望?请描述您希望得到的支持和机会。”问题,答案的平均长度为9.16个字符,中位数为7个字符。答案长度从1个字符到58个字符不等。

\item 对于“您对于提高人工智能教育质量和实践教学的其他意见和建议是什么?”问题,答案的平均长度为6.74个字符,中位数为3个字符。答案长度从1个字符到233个字符不等。

\end{itemize}

基于上述分析:

\begin{itemize}
\item \textbf{认真回答的标准}:如果回答时间合理(不低于中位数)且开放式问题的答案长度超过平均值,可以认为这是一次认真的回答。
\item \textbf{可能敷衍的回答}:回答时间极短(远低于中位数)或开放式问题的答案长度极短(接近最小值)的情况可能表明回答是敷衍的。
\end{itemize}

考虑到回答时间和答案长度的分布,进一步分析具体的回答数据,以识别哪些回答可能是认真的,哪些可能是敷衍的。需要注意的是,这种方法虽然可以提供一定的指导,但仍有可能出现误判,因为每个人的回答习惯和速度可能不同。

Q:
汇总结果,并作可视化处理。

A:
如图,回答评估结果的分布如下:
\begin{itemize}
\item 认真回答:38 份,显示为\textbf{蓝色条形}。
\item 可能敷衍:55 份,显示为\textbf{绿色条形}。
\item 普通回答:107 份,显示为\textbf{红色条形}。
\end{itemize}

根据分析结果,对于这 200 份调查问卷的回答:
\begin{itemize}
\item \textbf{38 份被评估为“认真回答”},表明这些回答在回答时间和开放式问题答
案长度上都显示出较高的认真程度。

\item \textbf{55 份被评估为“可能敷衍”},这些回答的回答时间较短或开放式问题的
答案长度较短,可能表明这些回答没有充分考虑问题。

\item \textbf{107 份被评为“普通回答”},这些回答可能没有明显的标志表明它们是特别认真或特别敷衍的。 
\end{itemize}
这个汇总结果提供了一个大致的指导,帮助识别哪些回答可能是更加认真和投入的,哪些可能没有充分考虑问题。这可以帮助进一步分析数据,尤其是在处理开放式反馈时,更加关注那些提供了详细信息和深入见解的回答。